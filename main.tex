% 	This template is  MIT licensed.

% 	Basic file to demonstrate the usage of this LaTeX template.
% 	You can build your own paper/thesis on top of this file.
% 	Simply adjust the document class and all metadata and start working.
%
\documentclass[
	language=english, % set to english or german
	type=bachelor, % set to bachelor, master or seminar
]{isthesis}

% Graphics rendering using TikZ
% See: https://en.wikibooks.org/wiki/LaTeX/PGF/TikZ

\usepackage{tikz}
% \usepackage{graphicx}
% \usepackage{caption}
% Include required TikZ libraries here, some exemplary libraries are pre-included
\usetikzlibrary{calc}
\usetikzlibrary{matrix}
\usetikzlibrary{positioning}
\usetikzlibrary{shapes.geometric}

%Add your library here
\addbibresource{library.bib}

% Import acronyms
% \newacronym[longplural={<long plural>}, shortplural={<short plural>}]{<label>}{<short>}{<long>}
% 	label = is the unique identifier and sort key for the acronym, can be the same as <short>
%	short = is the abbreviation or acronym
%	short plural (optional) = is the plural of the abbreviation or acronym
%	long = is the long form of the acronym, this will appear in the list of abbreviations
%	long plural (optional) = is the long plural form of the abbreviation or acronym

% \newacronym[shortplural={KMUen}, longplural={Kleine und Mittlere Unternehmen}]{kmu}{KMU}{Kleines und Mittleres Unternehmen}
% \newacronym{CD}{CD}{Corporate Design}
% \newacronym{SQL}{SQL}{Structured Query Language}
% \newacronym{FAU}{FAU}{Friedrich-Alexander-Universit\"at Erlangen-N\"urnberg}
% \newacronym{BPM}{BPM}{Business Process Management}
% \newacronym{npm}{NPM}{Node Package Manager}
% \newacronym{diss}{DISS}{Digital Industrial Service System}
\newacronym{DCF}{DCF}{Dispersion Compensating Fiber}
\newacronym{GVD}{GVD}{Group Velocity Dispersion}
\newacronym{NLSE}{NLSE}{Nonlinear Schrödinger Equation}
\newacronym{SPM}{SPM}{Self Phase Modulation}
\newacronym{SSFM}{SSFM}{Split-Step Fourier Method}
\newacronym{FFT}{FFT}{Fast Fourier Transform}
\newacronym{DFT}{DFT}{Discrete Fourier Transform}


% Import symbols
\input{config/symbols}

% Import custom commands
% If you want to define custom commands, please do so here
\DeclareUnicodeCharacter{03B1}{$\alpha $}
\DeclareUnicodeCharacter{03B2}{$\beta$}
\DeclareUnicodeCharacter{03B3}{$\gamma$}
\DeclareUnicodeCharacter{03B4}{$\delta$}
\DeclareUnicodeCharacter{03B6}{$\zeta $}
\DeclareUnicodeCharacter{03B7}{$\eta $}
\DeclareUnicodeCharacter{03B8}{$\theta $}
\DeclareUnicodeCharacter{03BA}{$\kappa $}
\DeclareUnicodeCharacter{03BB}{$\lambda $}
\DeclareUnicodeCharacter{03BC}{$\mu $}
\DeclareUnicodeCharacter{03BE}{$\xi $}
\DeclareUnicodeCharacter{03C0}{$\pi $}
\DeclareUnicodeCharacter{03C6}{$\phi $}
\DeclareUnicodeCharacter{03C9}{$\omega $}



% Document meta information
\isthesis{
    title={Development of a web-oriented code modeling the propagation of ultrashort pulses in optical fibre},
    author-name={Hernan Alberto Aguilera Abril}, % Separate multiple authors with commas
    author-email={hernan.aguilera@fau.de},
    %author-phone={+49 1776300853}, % Use international numbers format
    author-matriculation={Matrikelnummer: 22710623},
    % author-address={Erwin Rommel Str. 53},
    % author-zip={91058},
    author-city={Erlangen},
    principal-supervisor={Prof. Dr.-Ing. Bernhard Schmauss}, % This must be a professor
    associate-supervisor={Prof. Dr. Nicolas Joly}, % This is your main supervisor, i.e., a post doc or doctoral student
    tutor-supervisor={}, % If required, define an additional supervisor resp. tutor here
    group={Lehrstuhl für Hochfrequenztechnik},
    group-institute={Friedrich-Alexander University of Erlangen-Nuremberg},
    %studies={B.Sc. Elektrotechnik – Elektronik – Informationstechnik}, %your field of studies, i.e. Wirtschaftsinformatik or International Information Systems
    %
    %associate-group={}, % When the thesis is done in cooperation with another chair, add it here
    %associate-group-institute={}, % add cooperating institute or university here
    seminar={Photonik}, % The title of your seminar
    submission-date={2021-10-04} % The date you handed in your document: Format yyyy-mm-dd
    %primary-logo={}, % Uses the FAU logo by default
    %primary-logo-height={}, % Uses 16mm as default height
    %secondary-logo={}, % Logo of the secondary institution (cooperating chair/university), USES Faculty logo by default
    %secondary-logo-height={} % Uses 16mm as default height
}


\begin{document}
    % Title page
    \newcounter{savepage}
    \maketitle

	% Quote
    % You can put an optional quote page in front of your content
    %   \quotepage[author={Arthur C. Clarke}]{
    %   	        Any sufficiently advanced technology is indistinguishable from magic.
    %   }
    
    % Table of contents
    \tableofcontents

    % List of figures (if you have figures)
    \listoffigures

    % List of tables (if you have tables)
    %\listoftables
    
    % List of listings (if you have listings)
	%\lstlistoflistings

    % List of abbreviations (if you use acronyms)
    \listofabbreviations

    % List of symbols (if you use symbols)
    %\listofsymbols
	
	% Abstract
	%
	% Comment out this part, if you don't require an abstract
	\begin{abstract}
	    % Add your abstract here:
		\lipsum[1]
		%la  importancia de las comunicaciones con fibra optica ha venido aumentando en los últimos anos debido a su alta quality (bajas pérdidas y hablar mierda de eso), así mismo, el estudio de las comunicaciones opticas ha venido en aumento, por ello, es importante que los estudiantes entiendan a profundidad los conceptos que encierran a las comunicaciones ópticas, ya sea desde la práctica o desde un punto de vista físico. Diversas herramientas pueden ser utilizadas en el aprendizaje de los estudiantes, entre ellas estan las aplicaciones web, las cuales, debido a su fácil acceso y amplio desarrollo, permiten a los estudiantes acceder a la información que requieran cuando la requieran. Se desarrolló entonces una aplicación web implementando un código en python, capaz de solucionar las diversas ecuaciones presentadas en el libro AGRAVAL, displaying los diversos fenómenos que afectan las comunicaciones opticas no lineares, tales como la dispersion y las no linearidades como es el caso del SPM. Primero fueron tratados los efectos por separado (en un principio la dispersion y luego las no linearidades), luego, con la ayuda del SSFM para la solución de la NLSE, se pudo ver el efecto de ambos fenómenos actuando al mismo tiempo en la fibra optica. Finalmente se deploy una web app capaz de solucionar de forma rápida la NLSE con el cambio de ciertos parámetros y presentar los cambios / broadening tanto en el tiempo como en la frecuencia dependiendo el caso. 
	\end{abstract}
	
	% storing the last pagenumber
    \setcounter{savepage}{\value{page}}
    
    
    % Content
    \begin{content}
        % Add your content files:
		\chapter{Introduction}
% Objective:
% Develop a tool for the future students who will study nonlinear optics in fibres so that they can appreciate more easily the different physical processes happening to a short pulse propagating in a dispersive nonlinear medium. These include the influence of the dispersion (second and third order), the self-phase modulation, the emission of dispersive wave, the soliton dynamics. Eventually, the code will be released as a web-interface for the future students to directly modify the physical parameters of the pulse, but also of the fibre.**

The actual state of communications leads to new challenges in the field of optical systems, which causes an increased interest from students in optical communications. Nonlinear optics in fibers involve different physical processes while a short pulse propagates in a dispersive nonlinear medium. 

Despite the disturbing character of the nonlinear effects, e.g. crosstalk or increased bit error rates; one can take advantage of them in some applications due to their importance on telecommunications. Combinations of these effects with optical filters and interferometers can originate "purely optical regeneration",  or, together with the dispersion, solitons can be formed \cite{rein}.


Effects like the influence of second or third-order dispersion, self-phase modulation, the emission of a dispersive wave, or the soliton dynamics are some of the topics of this thesis. The main objective is to use the existing theory and create a code capable of simulating these effects and then deploy it online to provide an accessible way for students (or people interested in understanding the concepts surrounding nonlinear optics) to recapitulate the theory seen in the lecture. The third chapter explains the development and deployment of a tool using the created code. \textbf{**ADD Method of feedback**} 

At the same time, it is pertinent to understand some of the necessities of the students who will be the final users of the tool created. This interactive service should be able to adapt itself to be a good add-on and aid the professors with their explanations throughout the semester. Hence those who take the lectures can explore what usually is on the blackboard or the books and interact with plots and solutions of equations going further than just replacing values in an equation and waiting for it to be correct. Here is when Interactive Learning and its advantages obtain relevance and thus bestow meaning to the core of this project.


		%\input{sections/01-5_litreview}
		\chapter{Theoretical Framework}

This chapter reviews the theoretical base to model ultrashort-pulse propagation in fibers. Spectral and time broadening are fundamental aspects of nonlinear fiber optics. Understanding their basis and how they affect the propagation of an electromagnetic wave is an essential factor to both develop the code and test its effectiveness. Providing an overview of the equations describing the light propagation in a dispersive, nonlinear medium gives the principles to see what is solved with the back-end code at further stages in this thesis project.




\section{Pulse propagation in fibers}

Pulse propagation in fibers

Loss and dispersion originated by interactions of the electromagnetic field with the atoms of the medium, and frequency dependence of the refractive index affect the propagation of an electromagnetic wave through this medium \citep{dudley_taylor_2010}. Shape and spectrum of short pulses with widths between approx. 10ns and 10fs propagating in a fiber are affected by nonlinear and dispersive effects \citep{AgrawalBook}.  

    \subsection{Pulse evolution inside a single-mode fiber:}
        The slowly varying amplitude \emph{A(z,t)} can be described by the equation:
        \begin{equation}\label{eq_a}
        \begin{split}
\frac{\partial A}{\partial z}&= \overbrace{-\frac{\alpha}{2} - \beta_1\frac{\partial A}{\partial t} - \frac{i \beta_2}{2} \frac{\partial^2 A}{\partial t^2} +\frac{\beta_3}{6} \frac{\partial^3 A}{\partial t^3}}^{\mathbf{linear \ effects}}\\  
            & +\underbrace{i \gamma \left( 1 + \frac{i}{\omega_0} \frac{\partial}{\partial t} \right) \left( A(z,t) \int_{-\infty}^{\infty} R(t') \left|A(z, t-t') \right|^2 \ dt'  \right)}_{\mathbf{nonlinear \ effects}}
        \end{split}
        \end{equation}

    where: 
    \begin{itemize}
        \item t is the time variable [s]
        \item z is the space variable [m]
        \item $alpha$ is the attenuation coefficient [1/m]
        \item $beta_n$ are propagation constant coefficients $[s^n/m]$
        \item $\gamma$ is the nonlinear parameter [1/Wm]
        
            \begin{equation}
                \gamma = \frac{n_2 \omega_0}{c A_{eff}},
                \label{eq_gamma}
            \end{equation}
        with $n_2 \ [m^2/W]$ is the nonlinear refractive index, and $A_{eff}$  is the  effective core area.
        %or  n2 is the nonlinear-index coefficient. 
        \item R(t) [1/s] is the response function that includes electronic and vibrational contributions and is approximated by 
        
        \begin{equation}\label{eq_rt}
            R(t) = (1- f_R)\delta(t) + f_R h_R(t),
        \end{equation}
        where $f_R$ and $h_R(t)$ are the fractional contribution of the delayed Raman response and the Raman response function, respectively. Conforming to \cite{dudley_taylor_2010}, 
        \begin{equation}\label{eq_hr}
            h_R = \frac{\tau^2_1+\tau^2_2}{\tau_1\tau^2_2} exp(-t/\tau_2)sin(t/\tau_1)\Theta(t),
        \end{equation}
        here $\delta(t)$ and $\Theta(t)$ are the Delta Dirac and the Heaviside step functions, respectively; $f_R = 0.18$,  $\tau_1 =12.2 fs$, and $\tau_2 = 32 fs$.
         
         
        \item $\omega_0$  is the center angular frequency
    
    \end{itemize}
   
    One can asume that the evolution of the envelope pulse is slow, and remembering that the group velocity is defined by 
    
    \begin{equation}\label{eq_vg}
        v_g \equiv \frac{1}{\beta_1}, 
    \end{equation}
    the pulse can be moved at the group velocity using the reference frame 
    \begin{equation}\label{eq_t}
        T = t - \frac{z}{v_g}. 
    \end{equation}
    
    Thus, equation \eqref{eq_a} can be defined as 
    
    \begin{equation}\label{eq_b}
        \frac{\partial A}{\partial z}=-\frac{\alpha}{2} - \frac{i \beta_2}{2} \frac{\partial^2 A}{\partial T^2} +\frac{\beta_3}{6} \frac{\partial^3 A}{\partial T^3} +i \gamma   \left(A\left|A\right|^2+ \frac{i}{\omega_0} \frac{\partial}{\partial T} (A\left|A \right|^2)- A \ T_R \frac{\partial \left|A \right|^2}{\partial T} \right),
        \end{equation}
    where the first moment of the nonlinear (Raman) response is
    \begin{equation}\label{eq_TR}
        T_R = f_R\int_{0}^{\infty} t \ h_R(t) \ dt.
    \end{equation}

        
    \subsection{NLSE}
        According to \citep{AgrawalBook}, the equation  \eqref{eq_b} for the slowly varying envelope can be simplified for pulses of width $T_0 > 5 ps$ to
        \begin{equation}
                \frac{\partial A}{\partial z} = \frac{\alpha}{2}A-j \frac{\beta_2}{2}\frac{\partial^2A}{\partial T^2}+j\gamma|A|^2 A,
                \label{eq_nlse}
            \end{equation}
            \ \\
       It is noteworthy that the contribution of the third-dispersion term should be considered if $\beta_2 \approx 0$ (zero-dispersion region). Equation \eqref{eq_nlse} is known as the nonlinear Schrödinger equation (NLSE).


    \subsection{FFT}
    
    \section{Modeling the short  pulse propagation}
    In order to have an easier understanding of the physical processes that are involved in the propagation of short pulses in a medium, it will be first necessary to focus on the dispersion that affects this medium. The purpose of this section is to consider the fiber as a linear optical medium while studying the pulse-propagation problem, thus, it is possible to inspect the effect of GVD under certain circumstances where it dominates over the nonlinearities \citep{ AgrawalBook}. A further objective of this thesis is to implement a code capable of illustrate the Dispersion-induced broadening of optical pulses for both Gaussian and Hyperbolic-Secant (Sech) shapes.

        
  
            \begin{equation}\label{eq_A0}
                A(z,\tau ) = \sqrt{P_0}e^{\frac{-\alpha z}{2}} U(z,\tau)
            \end{equation}
            with
            \begin{equation}
                \tau =  \frac{T}{T_0} = \frac{t-z/v_g}{T_0} \qquad
            \end{equation}
        
        It is important to define relative lengths in order to understand the effect of the dispersion and/or nonlinearities on a pulse propagation.
            \ \\
            \ \\
            \begin{equation}
                L_D = \frac{T^2_0}{|\beta_2|} \qquad and \qquad L_{NL} = \frac{1}{\gamma P_0}
            \end{equation}
 
        
        \subsection{Modeling the effect of GVD}
        
        Supposing a fiber length L such as $L \sim L_D$ and $L << L_{NL}$ and setting $\gamma = 0$: 
            \begin{equation}
                \frac{\partial U}{\partial z} = -j\frac{\beta_2}{2}\frac{\partial^2U}{\partial T^2}
            \end{equation}
            \ \\
            \ \\
            \begin{equation}
                U(z,T) = FFT \left[ \widetilde{U}(0,\omega) \ exp\{j\frac{\beta_2}{2}\omega^2z\} \right]
            \end{equation}
            \ \\
            \begin{equation}
                \widetilde{U}(0,\omega)  = FFT^{-1} \left[ U(0,T)\right]
            \end{equation}
        
        
        \begin{equation}
                U(0,T) = 
                \begin{cases}
                    exp \left\{ -\frac{T}{2T^2_0} \right\} \qquad \text{for Gaussian-shaped pulses}  \\
                    sech\left( \frac{T}{T_0}\right) exp \left\{ -\frac{iCT^2}{2T^2_0} \right\} \qquad \text{for Hyperbolic-Secant pulses}  \\
                \end{cases}
            \end{equation}
            \begin{center}
                with $C =$  initial Chirp.
            \end{center}
            Having these equations and using \href{https://numpy.org/}{Numpy}, was written a code to plot the pulse evolution. This code allows the user to change values like $\beta_2$ to see how the dispersion causes a pulse broadening.
        
        
        \subsection{Modeling the effect of SPM}
        
        Now supposing a fiber length L such as $L \sim L_{NL}$ and $L << L_D$ and setting $\beta_2 = 0$, equation \eqref{nlse} becomes: 
            \begin{equation}
                \frac{\partial U}{\partial z} = j\frac{e^{-\alpha z}}{L_{NL}}|U|^2 U
            \end{equation}
            %seite 98
            The phase shift can be defined as 
            \begin{equation}
                \phi_{NL}(L,T) = |U(0,T)|^2 (L_{eff}/L_{NL})
            \end{equation}
            with 
            \begin{equation}
                L_{eff} = \frac{1-e^{-\alpha L}}{\alpha}
                \qquad if \quad alpha = 0 \text{, then} \quad L_{eff} = L
            \end{equation}
            
            That allow us to introduce the SPM-induced chirp as
            
            \begin{equation}
                \delta \omega(T) = -\frac{\partial \phi_{NL}}{\partial T} = -\left( \frac{L_{eff}}{L_{NL}} \right) \frac{\partial }{T} |U(0,T)|^2
                \label{deltaomega}
            \end{equation}
        
        
         For a super-Gaussian pulse 
            \begin{equation}
                \delta \omega(T) = \frac{2m L_{eff}}{T_0 L_{NL}}\left( \frac{T}{T_0}\right)^{2m-1}  exp\left\{ -\left( \frac{T}{T_0}\right)^{2m}   \right\}
            \end{equation}
        
        \subsection{Solution of the NLSE}
            \subsubsection{Dudley}
            \subsubsection{SSFM}
            
        \subsection{Other nonlinear effects}


%Write eq for propagation and explain LD vs LNL Bereiche ->  GVD only -> U(0,T) Gauss- und Sech-Pulse -> wie wird das Programm durchgeführt? Ergebnisse mithilfe Matplotlibs, und Darstelung in einer Web-App 
		\chapter{Creation of the Web-tool}

 \section{Modeling the short  pulse propagation}
    In order to have an easier understanding of the physical processes that are involved in the propagation of short pulses in a medium, it will be first necessary to focus on the dispersion that affects this medium. The purpose of this section is to consider the fiber as a linear optical medium while studying the pulse-propagation problem, thus, it is possible to inspect the effect of GVD under certain circumstances where it dominates over the nonlinearities \citep{ AgrawalBook}. A further objective of this thesis is to implement a code capable of illustrate the Dispersion-induced broadening of optical pulses for both Gaussian and Hyperbolic-Secant (Sech) shapes.

        
  
            \begin{equation}\label{eq_A0}
                A(z,\tau ) = \sqrt{P_0}e^{\frac{-\alpha z}{2}} U(z,\tau)
            \end{equation}
            with
            \begin{equation}
                \tau =  \frac{T}{T_0} = \frac{t-z/v_g}{T_0} \qquad
            \end{equation}
        
        It is important to define relative lengths in order to understand the effect of the dispersion and/or nonlinearities on a pulse propagation.
            \ \\
            \ \\
            \begin{equation}
                L_D = \frac{T^2_0}{|\beta_2|} \qquad and \qquad L_{NL} = \frac{1}{\gamma P_0}
            \end{equation}
 
        
        \subsection{Modeling the effect of GVD}
   
 
            
        Supposing a fiber length L such as $L \sim L_D$ and $L << L_{NL}$ and setting $\gamma = 0$: 
            \begin{equation}
                \frac{\partial U}{\partial z} = -j\frac{\beta_2}{2}\frac{\partial^2U}{\partial T^2}
            \end{equation}
            \ \\
            \ \\
            \begin{equation}
                U(z,T) = FFT \left[ \tilde{U}(0,\omega) \ exp\{j\frac{\beta_2}{2}\omega^2z\} \right]
            \end{equation}
            \ \\
            \begin{equation}
                \tilde{U}(0,\omega)  = FFT^{-1} \left[ U(0,T)\right]
            \end{equation}
        
        
        \begin{equation}
                U(0,T) = 
                \begin{cases}
                    exp \left\{ -\frac{T}{2T^2_0} \right\} \qquad \text{for Gaussian-shaped pulses}  \\
                    sech\left( \frac{T}{T_0}\right) exp \left\{ -\frac{iCT^2}{2T^2_0} \right\} \qquad \text{for Hyperbolic-Secant pulses}  \\
                \end{cases}
            \end{equation}
            \begin{center}
                with $C =$  initial Chirp.
            \end{center}
            Having these equations and using \href{https://numpy.org/}{Numpy}, was written a code to plot the pulse evolution. This code allows the user to change values like $\beta_2$ to see how the dispersion causes a pulse broadening.
        
        
        \subsection{Modeling the effect of SPM}
        
        Now supposing a fiber length L such as $L \sim L_{NL}$ and $L << L_D$ and setting $\beta_2 = 0$, equation \eqref{eq_nlse} becomes: 
            \begin{equation}
                \frac{\partial U}{\partial z} = j\frac{e^{-\alpha z}}{L_{NL}}|U|^2 U
            \end{equation}
            %seite 98
            The phase shift can be defined as 
            \begin{equation}
                \phi_{NL}(L,T) = |U(0,T)|^2 (L_{eff}/L_{NL})
            \end{equation}
            with 
            \begin{equation}
                L_{eff} = \frac{1-e^{-\alpha L}}{\alpha}
                \qquad if \quad alpha = 0 \text{, then} \quad L_{eff} = L
            \end{equation}
            
            That allow us to introduce the SPM-induced chirp as
            
            \begin{equation}
                \delta \omega(T) = -\frac{\partial \phi_{NL}}{\partial T} = -\left( \frac{L_{eff}}{L_{NL}} \right) \frac{\partial }{T} |U(0,T)|^2
                \label{deltaomega}
            \end{equation}
        
        
         For a super-Gaussian pulse 
            \begin{equation}
                \delta \omega(T) = \frac{2m L_{eff}}{T_0 L_{NL}}\left( \frac{T}{T_0}\right)^{2m-1}  exp\left\{ -\left( \frac{T}{T_0}\right)^{2m}   \right\}
            \end{equation}
        
        \subsection{Solution of the NLSE}
            \subsubsection{Dudley}
            \subsubsection{SSFM}


\section{GVD and SPM effects}
    \subsection{GVD code for python}
    \subsection{SPM code for python}
        \subsection{mid-point}
    \subsection{NLSE}
        \subsubsection{SSFM}
		\chapter{Results}
    At the early stages of testing and debugging, the pulse class (called in the Dash's app to plot and update with the callbacks) had to be necessarily declared as global. The global definition of this variable allowed us to share the results of the propagation's computation with the callback in charge of plotting the envelope without needing to trigger a new calculation for each value of z. That means that we have to treat the pulse envelope's plots and the heatmaps separately to avoid slower interactions in the app. This use of global variables is very effective while working locally, even though it led to problems when the app is used on different devices or browsers, affecting the results of all users at the same time.

According to Dash's documentation\footnote{\href{https://dash.plotly.com/sharing-data-between-callbacks}{Sharing Data Between Callbacks}}, the information between callbacks can (and should) be carried by using the Store component. After implementing this fix on the app, tests using a local server with two devices (local machine using two different web-browsers and a smartphone) showed that the problem didn't persist.
    \section{Deployment of the web-tool}
        \subsection{Heroku (Free servers)}
        \subsection{Prof. Joly's Web}




\chapter{Conclusion}
    \section{Feedback from students}
    \section{Future work}
		%\input{sections/02_Elements}
		
    \end{content}
    
    \pagenumbering{Roman}
    \setcounter{page}{\numexpr\value{savepage}}

    % References
    \references{}
    
    % Appendix
     \begin{appendix}
        % In the appendices, use \section{} instead of \chapter{}
         \section{App's interface accessed via Google Chrome on an Android device}
 \label{Appendixapp}

        
        
        \begin{figure}[label={fig:herokugvd}, caption={GVD-App accesed via Google Chrome in an Android device.}]
        \centering
        \begin{tabular}[c]{cc}
        \centering
        \begin{subfigure}[b]{.53\textwidth}
		    \centering	
            \includegraphics[width=1\textwidth]{figures/chap4/android_gvd1.png}
            \caption{First part of the GVD-app.}
            \label{fig:herokugvd1}
        \end{subfigure}
        \hfill
        \begin{subfigure}[b]{.53\textwidth}
		    \centering	
            \includegraphics[width=1\textwidth]{figures/chap4/android_gvd2.png}
            \caption{Second part of the GVD-app.}
            \label{fig:herokugvd2}
        \end{subfigure}
        \end{tabular}
        \end{figure}
        
        \begin{figure}[label={fig:herokuspm}, caption={SPM-App accesed via Google Chrome in an Android device.}]
          %  \caption*{Source: Some Source}
        	\includegraphics[width=.8\textwidth]{figures/chap4/android_spm.png} 
        \end{figure}
        

    \begin{figure}[label={fig:herokunlse}, caption={NLSE-App accesed via Google Chrome in an Android device.}]
        \centering
        \begin{tabular}[c]{cc}
        \centering
        \begin{subfigure}[b]{.53\textwidth}
		    \centering	
            \includegraphics[width=1\textwidth]{figures/chap4/android_nlse1.png}
            \caption{First part of the NLSE-app.}
            \label{fig:herokunlse1}
        \end{subfigure}
        \hfill
        \begin{subfigure}[b]{.53\textwidth}
		    \centering	
            \includegraphics[width=1\textwidth]{figures/chap4/android_nlse2.png}
            \caption{Second part of the NLSE-app.}
            \label{fig:herokunlse2}
        \end{subfigure}
        \end{tabular}
        \end{figure}

         \section{Running the code}
\label{sec:initvariables}
It is necessary to install all the packages and download the elements of this project to run the code. One can find the required files on Github\footnote{\url{https://github.com/haaguileraa/NLSE}}, there is possible to either clone the repository or download the files on a ZIP file. The code was written using Python 3, install this programming language on the machine is therefore mandatory. 

Firstly, we should upgrade pip using:
\ \\

\begin{lstlisting}
python -m pip install --upgrade pip .
\end{lstlisting}

The next step is to install the packages by using the command line:
\ \\

\begin{lstlisting}
python -m pip install -r requirements.txt . 
\end{lstlisting}


After following this process, the computer should fulfill the prerequisites to run the file "index.py." Consequently, the Dash app for GVD is available (by default) using the desired web browser and opening the link \url{http://127.0.0.1:8050/apps/gvd}. One can access the SPM and NLSE apps by clicking the links on each layout, as well as by changing the path from "apps/gvd" to "apps/spm" and "apps/nlse" consecutively.

However, it is also possible to run the code using Matplotlib utilizing the file "prop\_matplot.py." Consequently, the user can change the parameters manually directly into the code; nevertheless, its use requires a higher level of expertise using Python.

%\end{footnotesize}
         
     \end{appendix}




    % Declaration of authorship
    % \authorshipstatement[pagenumbering=false]
    \authorshipstatement[pagenumbering=true]
    % \authorshipstatement[pagenumbering=only]
    
    % Consent form for use of plagiarism detection software
    % Not yet required
    % \consentform[pagenumbering=false]
    % \consentform[pagenumbering=true]
    % \consentform[pagenumbering=only]
    
    % Bonus: Wordcount
    % cd %FOLDER WHERE THE .tex FILES ARE IN %
    % clear
    % texcount -total -q -col -sum *.tex
    
\end{document}