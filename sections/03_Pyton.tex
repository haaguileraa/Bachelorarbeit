\chapter{Creation of the Web-tool}

 \section{Modeling the short  pulse propagation}
    In order to have an easier understanding of the physical processes that are involved in the propagation of short pulses in a medium, it will be first necessary to focus on the dispersion that affects this medium. The purpose of this section is to consider the fiber as a linear optical medium while studying the pulse-propagation problem, thus, it is possible to inspect the effect of GVD under certain circumstances where it dominates over the nonlinearities \citep{ AgrawalBook}. A further objective of this thesis is to implement a code capable of illustrate the Dispersion-induced broadening of optical pulses for both Gaussian and Hyperbolic-Secant (Sech) shapes.

        
  
            \begin{equation}\label{eq_A0}
                A(z,\tau ) = \sqrt{P_0}e^{\frac{-\alpha z}{2}} U(z,\tau)
            \end{equation}
            with
            \begin{equation}
                \tau =  \frac{T}{T_0} = \frac{t-z/v_g}{T_0} \qquad
            \end{equation}
        
        It is important to define relative lengths in order to understand the effect of the dispersion and/or nonlinearities on a pulse propagation.
            \ \\
            \ \\
            \begin{equation}
                L_D = \frac{T^2_0}{|\beta_2|} \qquad and \qquad L_{NL} = \frac{1}{\gamma P_0}
            \end{equation}
 
        
        \subsection{Modeling the effect of GVD}
   
 
            
        Supposing a fiber length L such as $L \sim L_D$ and $L << L_{NL}$ and setting $\gamma = 0$: 
            \begin{equation}
                \frac{\partial U}{\partial z} = -j\frac{\beta_2}{2}\frac{\partial^2U}{\partial T^2}
            \end{equation}
            \ \\
            \ \\
            \begin{equation}
                U(z,T) = FFT \left[ \tilde{U}(0,\omega) \ exp\{j\frac{\beta_2}{2}\omega^2z\} \right]
            \end{equation}
            \ \\
            \begin{equation}
                \tilde{U}(0,\omega)  = FFT^{-1} \left[ U(0,T)\right]
            \end{equation}
        
        
        \begin{equation}
                U(0,T) = 
                \begin{cases}
                    exp \left\{ -\frac{T}{2T^2_0} \right\} \qquad \text{for Gaussian-shaped pulses}  \\
                    sech\left( \frac{T}{T_0}\right) exp \left\{ -\frac{iCT^2}{2T^2_0} \right\} \qquad \text{for Hyperbolic-Secant pulses}  \\
                \end{cases}
            \end{equation}
            \begin{center}
                with $C =$  initial Chirp.
            \end{center}
            Having these equations and using \href{https://numpy.org/}{Numpy}, was written a code to plot the pulse evolution. This code allows the user to change values like $\beta_2$ to see how the dispersion causes a pulse broadening.
        
        
        \subsection{Modeling the effect of SPM}
        
        Now supposing a fiber length L such as $L \sim L_{NL}$ and $L << L_D$ and setting $\beta_2 = 0$, equation \eqref{eq_nlse} becomes: 
            \begin{equation}
                \frac{\partial U}{\partial z} = j\frac{e^{-\alpha z}}{L_{NL}}|U|^2 U
            \end{equation}
            %seite 98
            The phase shift can be defined as 
            \begin{equation}
                \phi_{NL}(L,T) = |U(0,T)|^2 (L_{eff}/L_{NL})
            \end{equation}
            with 
            \begin{equation}
                L_{eff} = \frac{1-e^{-\alpha L}}{\alpha}
                \qquad if \quad alpha = 0 \text{, then} \quad L_{eff} = L
            \end{equation}
            
            That allow us to introduce the SPM-induced chirp as
            
            \begin{equation}
                \delta \omega(T) = -\frac{\partial \phi_{NL}}{\partial T} = -\left( \frac{L_{eff}}{L_{NL}} \right) \frac{\partial }{T} |U(0,T)|^2
                \label{deltaomega}
            \end{equation}
        
        
         For a super-Gaussian pulse 
            \begin{equation}
                \delta \omega(T) = \frac{2m L_{eff}}{T_0 L_{NL}}\left( \frac{T}{T_0}\right)^{2m-1}  exp\left\{ -\left( \frac{T}{T_0}\right)^{2m}   \right\}
            \end{equation}
        
        \subsection{Solution of the NLSE}
            \subsubsection{Dudley}
            \subsubsection{SSFM}


\section{GVD and SPM effects}
    \subsection{GVD code for python}
    \subsection{SPM code for python}
        \subsection{mid-point}
    \subsection{NLSE}
        \subsubsection{SSFM}