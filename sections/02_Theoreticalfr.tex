\chapter{Theoretical Framework}

This chapter reviews the theoretical base to model ultrashort-pulse propagation in fibers. Spectral and time broadening are fundamental aspects of nonlinear fiber optics. Understanding their basis and how they affect the propagation of an electromagnetic wave is an essential factor to both develop the code and test its effectiveness. Providing an overview of the equations describing the light propagation in a dispersive, nonlinear medium gives the principles to see what is solved with the back-end code at further stages in this thesis project.




\section{Pulse propagation in fibers}

Pulse propagation in fibers

Loss and dispersion originated by interactions of the electromagnetic field with the atoms of the medium, and frequency dependence of the refractive index affect the propagation of an electromagnetic wave through this medium \citep{dudley_taylor_2010}. Shape and spectrum of short pulses with widths between approx. 10ns and 10fs propagating in a fiber are affected by nonlinear and dispersive effects \citep{AgrawalBook}.  

    \subsection{Pulse evolution inside a single-mode fiber:}
        The slowly varying amplitude \emph{A(z,t)} can be described by the equation:
        \begin{equation}\label{eq_a}
        \begin{split}
\frac{\partial A}{\partial z}&= \overbrace{-\frac{\alpha}{2} - \beta_1\frac{\partial A}{\partial t} - \frac{i \beta_2}{2} \frac{\partial^2 A}{\partial t^2} +\frac{\beta_3}{6} \frac{\partial^3 A}{\partial t^3}}^{\mathbf{linear \ effects}}\\  
            & +\underbrace{i \gamma \left( 1 + \frac{i}{\omega_0} \frac{\partial}{\partial t} \right) \left( A(z,t) \int_{-\infty}^{\infty} R(t') \left|A(z, t-t') \right|^2 \ dt'  \right)}_{\mathbf{nonlinear \ effects}}
        \end{split}
        \end{equation}

    where: 
    \begin{itemize}
        \item t is the time variable [s]
        \item z is the space variable [m]
        \item $alpha$ is the attenuation coefficient [1/m]
        \item $beta_n$ are propagation constant coefficients $[s^n/m]$
        \item $\gamma$ is the nonlinear parameter [1/Wm]
        
            \begin{equation}
                \gamma = \frac{n_2 \omega_0}{c A_{eff}},
                \label{eq_gamma}
            \end{equation}
        with $n_2 \ [m^2/W]$ is the nonlinear refractive index, and $A_{eff}$  is the  effective core area.
        %or  n2 is the nonlinear-index coefficient. 
        \item R(t) [1/s] is the response function that includes electronic and vibrational contributions and is approximated by 
        
        \begin{equation}\label{eq_rt}
            R(t) = (1- f_R)\delta(t) + f_R h_R(t),
        \end{equation}
        where $f_R$ and $h_R(t)$ are the fractional contribution of the delayed Raman response and the Raman response function, respectively. Conforming to \cite{dudley_taylor_2010}, 
        \begin{equation}\label{eq_hr}
            h_R = \frac{\tau^2_1+\tau^2_2}{\tau_1\tau^2_2} exp(-t/\tau_2)sin(t/\tau_1)\Theta(t),
        \end{equation}
        here $\delta(t)$ and $\Theta(t)$ are the Delta Dirac and the Heaviside step functions, respectively; $f_R = 0.18$,  $\tau_1 =12.2 fs$, and $\tau_2 = 32 fs$.
         
         
        \item $\omega_0$  is the center angular frequency
    
    \end{itemize}
   
    One can asume that the evolution of the envelope pulse is slow, and remembering that the group velocity is defined by 
    
    \begin{equation}\label{eq_vg}
        v_g \equiv \frac{1}{\beta_1}, 
    \end{equation}
    the pulse can be moved at the group velocity using the reference frame 
    \begin{equation}\label{eq_t}
        T = t - \frac{z}{v_g}. 
    \end{equation}
    
    Thus, equation \eqref{eq_a} can be defined as 
    
    \begin{equation}\label{eq_b}
        \frac{\partial A}{\partial z}=-\frac{\alpha}{2} - \frac{i \beta_2}{2} \frac{\partial^2 A}{\partial T^2} +\frac{\beta_3}{6} \frac{\partial^3 A}{\partial T^3} +i \gamma   \left(A\left|A\right|^2+ \frac{i}{\omega_0} \frac{\partial}{\partial T} (A\left|A \right|^2)- A \ T_R \frac{\partial \left|A \right|^2}{\partial T} \right),
        \end{equation}
    where the first moment of the nonlinear (Raman) response is
    \begin{equation}\label{eq_TR}
        T_R = f_R\int_{0}^{\infty} t \ h_R(t) \ dt.
    \end{equation}

        
    \subsection{NLSE}
        According to \citep{AgrawalBook}, the equation  \eqref{eq_b} for the slowly varying envelope can be simplified for pulses of width $T_0 > 5 ps$ to
        \begin{equation}
                \frac{\partial A}{\partial z} = \frac{\alpha}{2}A-j \frac{\beta_2}{2}\frac{\partial^2A}{\partial T^2}+j\gamma|A|^2 A,
                \label{eq_nlse}
            \end{equation}
            \ \\
       It is noteworthy that the contribution of the third-dispersion term should be considered if $\beta_2 \approx 0$ (zero-dispersion region). Equation \eqref{eq_nlse} is known as the nonlinear Schrödinger equation (NLSE).


    \subsection{FFT}
    
    \section{Modeling the short  pulse propagation}
    In order to have an easier understanding of the physical processes that are involved in the propagation of short pulses in a medium, it will be first necessary to focus on the dispersion that affects this medium. The purpose of this section is to consider the fiber as a linear optical medium while studying the pulse-propagation problem, thus, it is possible to inspect the effect of GVD under certain circumstances where it dominates over the nonlinearities \citep{ AgrawalBook}. A further objective of this thesis is to implement a code capable of illustrate the Dispersion-induced broadening of optical pulses for both Gaussian and Hyperbolic-Secant (Sech) shapes.

        
  
            \begin{equation}\label{eq_A0}
                A(z,\tau ) = \sqrt{P_0}e^{\frac{-\alpha z}{2}} U(z,\tau)
            \end{equation}
            with
            \begin{equation}
                \tau =  \frac{T}{T_0} = \frac{t-z/v_g}{T_0} \qquad
            \end{equation}
        
        It is important to define relative lengths in order to understand the effect of the dispersion and/or nonlinearities on a pulse propagation.
            \ \\
            \ \\
            \begin{equation}
                L_D = \frac{T^2_0}{|\beta_2|} \qquad and \qquad L_{NL} = \frac{1}{\gamma P_0}
            \end{equation}
 
        
        \subsection{Modeling the effect of GVD}
        
        Supposing a fiber length L such as $L \sim L_D$ and $L << L_{NL}$ and setting $\gamma = 0$: 
            \begin{equation}
                \frac{\partial U}{\partial z} = -j\frac{\beta_2}{2}\frac{\partial^2U}{\partial T^2}
            \end{equation}
            \ \\
            \ \\
            \begin{equation}
                U(z,T) = FFT \left[ \widetilde{U}(0,\omega) \ exp\{j\frac{\beta_2}{2}\omega^2z\} \right]
            \end{equation}
            \ \\
            \begin{equation}
                \widetilde{U}(0,\omega)  = FFT^{-1} \left[ U(0,T)\right]
            \end{equation}
        
        
        \begin{equation}
                U(0,T) = 
                \begin{cases}
                    exp \left\{ -\frac{T}{2T^2_0} \right\} \qquad \text{for Gaussian-shaped pulses}  \\
                    sech\left( \frac{T}{T_0}\right) exp \left\{ -\frac{iCT^2}{2T^2_0} \right\} \qquad \text{for Hyperbolic-Secant pulses}  \\
                \end{cases}
            \end{equation}
            \begin{center}
                with $C =$  initial Chirp.
            \end{center}
            Having these equations and using \href{https://numpy.org/}{Numpy}, was written a code to plot the pulse evolution. This code allows the user to change values like $\beta_2$ to see how the dispersion causes a pulse broadening.
        
        
        \subsection{Modeling the effect of SPM}
        
        Now supposing a fiber length L such as $L \sim L_{NL}$ and $L << L_D$ and setting $\beta_2 = 0$, equation \eqref{nlse} becomes: 
            \begin{equation}
                \frac{\partial U}{\partial z} = j\frac{e^{-\alpha z}}{L_{NL}}|U|^2 U
            \end{equation}
            %seite 98
            The phase shift can be defined as 
            \begin{equation}
                \phi_{NL}(L,T) = |U(0,T)|^2 (L_{eff}/L_{NL})
            \end{equation}
            with 
            \begin{equation}
                L_{eff} = \frac{1-e^{-\alpha L}}{\alpha}
                \qquad if \quad alpha = 0 \text{, then} \quad L_{eff} = L
            \end{equation}
            
            That allow us to introduce the SPM-induced chirp as
            
            \begin{equation}
                \delta \omega(T) = -\frac{\partial \phi_{NL}}{\partial T} = -\left( \frac{L_{eff}}{L_{NL}} \right) \frac{\partial }{T} |U(0,T)|^2
                \label{deltaomega}
            \end{equation}
        
        
         For a super-Gaussian pulse 
            \begin{equation}
                \delta \omega(T) = \frac{2m L_{eff}}{T_0 L_{NL}}\left( \frac{T}{T_0}\right)^{2m-1}  exp\left\{ -\left( \frac{T}{T_0}\right)^{2m}   \right\}
            \end{equation}
        
        \subsection{Solution of the NLSE}
            \subsubsection{Dudley}
            \subsubsection{SSFM}
            
        \subsection{Other nonlinear effects}


%Write eq for propagation and explain LD vs LNL Bereiche ->  GVD only -> U(0,T) Gauss- und Sech-Pulse -> wie wird das Programm durchgeführt? Ergebnisse mithilfe Matplotlibs, und Darstelung in einer Web-App 