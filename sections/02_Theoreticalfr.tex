\chapter{Theoretical Framework}

This chapter reviews the theoretical base to model ultrashort-pulse propagation in fibers. Spectral and time broadening are fundamental aspects of nonlinear fiber optics. Understanding their basis and how they affect the propagation of an electromagnetic wave is an essential factor to both develop the code and test its effectiveness. Providing an overview of the equations describing the light propagation in a dispersive, nonlinear medium gives the principles to see what is solved with the back-end code at further stages in this thesis project.




\section{Pulse propagation in fibers}

Pulse propagation in fibers

Loss and dispersion originated by interactions of the electromagnetic field with the atoms of the medium, and frequency dependence of the refractive index affect the propagation of an electromagnetic wave through this medium \citep{dudley_taylor_2010}. Shape and spectrum of short pulses with widths between approx. 10ns and 10fs propagating in a fiber are affected by nonlinear and dispersive effects \citep{AgrawalBook}.  

    \subsection{Pulse evolution inside a single-mode fiber:}
        The slowly varying amplitude \emph{A(z,t)} can be described by the equation:
        \begin{equation}\label{eq_a}
        \begin{split}
\frac{\partial A}{\partial z}&= \overbrace{-\frac{\alpha}{2} - \beta_1\frac{\partial A}{\partial t} - \frac{i \beta_2}{2} \frac{\partial^2 A}{\partial t^2} +\frac{\beta_3}{6} \frac{\partial^3 A}{\partial t^3}}^{\mathbf{linear \ effects}}\\  
            & +\underbrace{i \gamma \left( 1 + \frac{i}{\omega_0} \frac{\partial}{\partial t} \right) \left( A(z,t) \int_{-\infty}^{\infty} R(t') \left|A(z, t-t') \right|^2 \ dt'  \right)}_{\mathbf{nonlinear \ effects}}
        \end{split}
        \end{equation}

    where: 
\begin{itemize}
    \item t is the time variable [s]
    \item z is the space variable [m]
    \item $alpha$ is the attenuation coefficient [1/m]
    \item $beta_n$ are propagation constant coefficients $[s^n/m]$
    \item $\gamma$ is the nonlinear parameter [1/Wm]
    \item R(t) is the response function that includes electronic and vibrational contributions [1/s]
        \begin{equation}
            \gamma = \frac{n_2 \omega_0}{c A_{eff}}
            \label{eq_gamma}
        \end{equation}
    with $n_2 \ [m^2/W]$ is the nonlinear refractive index, and $A_{eff}$  is the  effective core area.
    %or  n2 is the nonlinear-index coefficient. 
    \item 
\end{itemize}

    \section{Modeling the short  pulse propagation}
    In order to have an easier understanding of the physical processes that are involved in the propagation of short pulses in a medium, it will be first necessary to focus on the dispersion that affects this medium. The purpose of this section is to consider the fiber as a linear optical medium while studying the pulse-propagation problem, thus, it is possible to inspect the effect of GVD under certain circumstances where it dominates over the nonlinearities \citep{ AgrawalBook}. A further objective of this thesis is to implement a code capable of illustrate the Dispersion-induced broadening of optical pulses for both Gaussian and Hyperbolic-Secant (Sech) shapes.
    
        \subsection{NLSE}
        
        \subsection{Modeling the effect of GVD}
        \subsection{Modeling the effect of SPM}
        
        \subsection{Solution of the NLSE}
            \subsubsection{Dudley}
            \subsubsection{SSFM}
            
        \subsection{Other nonlinear effects}


%Write eq for propagation and explain LD vs LNL Bereiche ->  GVD only -> U(0,T) Gauss- und Sech-Pulse -> wie wird das Programm durchgeführt? Ergebnisse mithilfe Matplotlibs, und Darstelung in einer Web-App 