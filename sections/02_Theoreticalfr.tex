\chapter{Theoretical Framework}

This chapter reviews the theoretical base to model ultrashort-pulse propagation in fibers. Spectral and time broadening are fundamental aspects of nonlinear fiber optics. Understanding their basis and how they affect the propagation of an electromagnetic wave is an essential factor to both develop the code and test its effectiveness. Providing an overview of the equations describing the light propagation in a dispersive, nonlinear medium gives the principles to see what is solved with the back-end code at further stages in this thesis project.

\section{Eigenvalue problem}

\subsection{Planar Waveguide}
  
Figure 2 shows a planar waveguide consisting of two materials with refractive indices $n_1$ for the core and $n_2$ for the cladding. This waveguide is infinite in the y-direction, has a thickness $d$ (which satisfies that $d >> \lambda$), and fulfills the total internal reflection condition at the boundary.  Its transverse propagation constant is 

            \begin{equation}
                h=\sqrt{k_o^2n_1^2-\beta^2},
                \label{eq_h}
            \end{equation}
 where $\beta$ and $h$ are for the core region. For the cladding region, we have the longitudinal propagation constant $\kappa$

             \begin{equation}
                \kappa=\sqrt{\beta^2-k_o^2n_2^2}.
                \label{gam}
            \end{equation}
 
 Here $k_0 = \omega_0/c$ is the wavenumber in vacuum.

We can include U, W, and V

            \begin{equation}
                U^2+W^2 = V^2 = k_o^2a^2(n_1^2-n_2^2) \
                \begin{cases}
                    U = a \times h \\
                    W = a \times \kappa
                \end{cases} 
                \label{Normv},
            \end{equation}
where V is the so-called waveguide parameter or normalized frequency. 
The TE mode is then given by 
         
            
            \begin{equation}
            \textbf{TE mode} \ \ W=
                \begin{cases}
                    U \tan(U) & \text{even mode}\\
                    -U cotan(U) & \text{odd mode}
                \end{cases}
                \label{Temode}
            \end{equation}

We can solve the eigenvalue problem graphically by plotting W as a function of U [Joly's lecture]. Figure 2 shows this approach where the circle has a radius of V.

We created a code at the Max-Plank institute [5] to show the students how a change of parameters can affect the plot and, thus, the solutions of the eigenvalue problem. This web tool aims to help the students understanding this problem, and figure 2 shows the interface. 



The code can also solve the problem for the optical fiber, and the students can change some more parameters like cladding and core material, see the Poynting vector or some of the cuts of the field. Figures 3 to 6 show this web. 


Heroku is a web that supports programming languages like Java, Node.js, Python, among others [heroku]. The codes for the planar waveguide and the optical fiber were deployed using its free service and are available on the internet via this link.


   


\section{Pulse propagation in fibers}

Pulse propagation in fibers

Loss and dispersion originated by interactions of the electromagnetic field with the atoms of the medium, and frequency dependence of the refractive index affect the propagation of an electromagnetic wave through this medium \citep{dudley_taylor_2010}. Shape and spectrum of short pulses with widths between approx. 10ns and 10fs propagating in a fiber are affected by nonlinear and dispersive effects \citep{AgrawalBook}.  

    \subsection{Pulse evolution inside a single-mode fiber:}
        The slowly varying amplitude \emph{A(z,t)} can be described by the equation:
        \begin{equation}\label{eq_a}
        \begin{split}
\frac{\partial A}{\partial z}&= \overbrace{-\frac{\alpha}{2} - \beta_1\frac{\partial A}{\partial t} - \frac{i \beta_2}{2} \frac{\partial^2 A}{\partial t^2} +\frac{\beta_3}{6} \frac{\partial^3 A}{\partial t^3}}^{\mathbf{linear \ effects}}\\  
            & +\underbrace{i \gamma \left( 1 + \frac{i}{\omega_0} \frac{\partial}{\partial t} \right) \left( A(z,t) \int_{-\infty}^{\infty} R(t') \left|A(z, t-t') \right|^2 \ dt'  \right)}_{\mathbf{nonlinear \ effects}}
        \end{split}
        \end{equation}
    %[A] = sqrt(W)
    where: 
    \begin{itemize}
        \item \textbf{t} is the time variable [s]
        \item \textbf{z} is the space variable [m]
        \item $\mathbf{\alpha}$ is the attenuation coefficient [1/m]
        \item $\mathbf{\beta_n}$ are propagation constant coefficients $[s^n/m]$
        \item $\mathbf{\gamma}$ is the nonlinear parameter [1/Wm]
        
            \begin{equation}
                \gamma = \frac{n_2 \omega_0}{c A_{eff}},
                \label{eq_gamma}
            \end{equation}
        with $\mathbf{n_2} \quad [m^2/W]$ which is the nonlinear refractive index, and $A_{eff} \quad [m^2]$ that  is the  effective core area.
        %or  n2 is the nonlinear-index coefficient. 
        \item \textbf{R(t)} [1/s] is the response function that includes electronic and vibrational contributions and is approximated by 
        
        \begin{equation}\label{eq_rt}
            R(t) = (1- f_R)\delta(t) + f_R h_R(t),
        \end{equation}
        where $f_R$ and $h_R(t)$ are the fractional contribution of the delayed Raman response and the Raman response function, respectively. Conforming to \cite{dudley_taylor_2010}, 
        \begin{equation}\label{eq_hr}
            h_R(t) = \frac{\tau^2_1+\tau^2_2}{\tau_1\tau^2_2} exp(-t/\tau_2)sin(t/\tau_1)\Theta(t),
        \end{equation}
        here $\delta(t)$ and $\Theta(t)$ are the Delta Dirac and the Heaviside step functions, respectively; $f_R = 0.18$,  $\tau_1 =12.2 fs$, and $\tau_2 = 32 fs$.
         
         
        \item $\omega_0$  is the center angular frequency
    
    \end{itemize}
   
    One can asume that the evolution of the envelope pulse is slow, and remembering that the group velocity is defined by 
    
    \begin{equation}\label{eq_vg}
        v_g \equiv \frac{1}{\beta_1}, 
    \end{equation}
    the pulse can be moved at the group velocity using the reference frame 
    \begin{equation}\label{eq_t}
        T = t - \frac{z}{v_g}. 
    \end{equation}
    
    Thus, equation \eqref{eq_a} can be defined as 
    
    \begin{equation}\label{eq_b}
        \frac{\partial A}{\partial z}=-\frac{\alpha}{2} - \frac{i \beta_2}{2} \frac{\partial^2 A}{\partial T^2} +\frac{\beta_3}{6} \frac{\partial^3 A}{\partial T^3} +i \gamma   \left(A\left|A\right|^2+ \frac{i}{\omega_0} \frac{\partial}{\partial T} (A\left|A \right|^2)- A \ T_R \frac{\partial \left|A \right|^2}{\partial T} \right),
        \end{equation}
    where the first moment of the nonlinear (Raman) response is
    \begin{equation}\label{eq_TR}
        T_R = f_R\int_{0}^{\infty} t \ h_R(t) \ dt.
    \end{equation}

        
    \subsection{NLSE}
        According to \citep{AgrawalBook}, the equation  \eqref{eq_b} for the slowly varying envelope can be simplified for pulses of width $T_0 > 5 ps$ to
        \begin{equation}
                \frac{\partial A}{\partial z} = \frac{\alpha}{2}A-j \frac{\beta_2}{2}\frac{\partial^2A}{\partial T^2}+j\gamma|A|^2 A,
                \label{eq_nlse}
            \end{equation}
            \ \\
       It is noteworthy that the contribution of the third-dispersion term should be considered if $\beta_2 \approx 0$ (zero-dispersion region). Equation \eqref{eq_nlse} is known as the nonlinear Schrödinger equation (NLSE).

           Following the convention used in \cite{AgrawalBook}, \cite{dudley_taylor_2010} , the continuous inverse Fourier transform of the slowly varying function of z is
        
        \begin{equation}\label{eq_acft}
            A(z,t) = \frac{1}{2*\pi} \int_{-\infty}^{\infty} \tilde{A}(z,\omega-\omega_0)\exp{[-i(\omega-\omega_0)t]} \ d\omega.
        \end{equation}
        
        
        The FFT is an algorithm capable of calculating the discrete Fourier transform faster \citep{Lynch2018}. The definition of \href{https://numpy.org/doc/stable/index.html}{Numpy} of the inverse DFT \cite{dft}: 
        \begin{equation}\label{eq_dft}
            a_n = \frac{1}{n}\sum_{k=0}^{n-1} A_k \ exp\left\{ 2\pi i \frac{mk}{n} \right\} \qquad m = 0,...,n-1,
        \end{equation}

        then it follows
        \begin{equation} \label{eq_deffft}
                A(z,T) = FFT \left[ \tilde{A}(z,\omega) \right].
            \end{equation}
            
            
    
   
        \subsection{Dispersion}
        
        
        
        \subsection{Other nonlinear effects}


%Write eq for propagation and explain LD vs LNL Bereiche ->  GVD only -> U(0,T) Gauss- und Sech-Pulse -> wie wird das Programm durchgeführt? Ergebnisse mithilfe Matplotlibs, und Darstelung in einer Web-App 