\chapter{Theoretical Framework}



    \section{Modeling the short  pulse propagation}
    In order to have an easier understanding of the physical processes that are involved in the propagation of short pulses in a medium, it will be first necessary to focus on the dispersion that affects this medium. The purpose of this chapter is to consider the fiber as a linear optical medium while studying the pulse-propagation problem, thus, it is possible to inspect the effect of GVD under certain circumstances where it dominates over the nonlinearities \citep{ AgrawalBook}. A further objective of this chapter is to implement a code capable of illustrate the Dispersion-induced broadening of optical pulses for both Gaussian and Hyperbolic-Secant (Sech) shapes.
    
        \subsection{NLSE}
        
        \subsection{Modeling the effect of GVD}
        \subsection{Modeling the effect of SPM}
        
        \subsection{Solution of the NLSE}
            \subsubsection{Dudley}
            \subsubsection{SSFM}
            
        \subsection{Other nonlinear effects}


%Write eq for propagation and explain LD vs LNL Bereiche ->  GVD only -> U(0,T) Gauss- und Sech-Pulse -> wie wird das Programm durchgeführt? Ergebnisse mithilfe Matplotlibs, und Darstelung in einer Web-App 