\chapter{Conclusions}
\begin{itemize}
    \item Using E-learning strategies can lead to an improvement of lectures and engagement of students; therefore, Professors can implement this web application to assist or even improve their lessons in the field of optical communications. 
    \item Handling the data and correctly working with the elements of this sort of program is evidently a crucial factor while building an interactive and didactic application. Similarly, designing the components and details is undoubtedly a turning point to facilitate the use of the software.
    
    \item \textbf{Hosting the server:} It seems likely to host the server at the Max Planck Institut or using fee-required services like Amazon AWS or DigitalOcean. However, such platforms require more technical support in contrast with Heroku, for example. Furthermore, server security has to be considered to avoid future issues.
\end{itemize}

To summarise, we are evaluating the course of action of proceeding with Heroku's pricing option. Owing to technical, financial, and administrative matters, it is a plausible idea to merge both Fiber Modes and Pulse evolution apps together; in other words, we will only have one program running under the same web service. As a result, the students will have access to both applications with a significantly higher speed of calculations, interactivity, and accessibility on the same site.
    
    
\section{Future work}


    We can still perform more code enhancements to make the program operating faster; for instance, changing default type arrays can lead to a size reduction of 50-75\%. This adjustment can contribute to the memory problem of web browsers or speed. It is then feasible to implement more features to the web, e.g., adding effects like SRS. More space to save data represents then the possibility of displaying more options and a plot of the phase, for example. However, the quality of results using this modification is yet to be proven.
    
    
    To conclude, the final users should test the different websites delivered by us. After a determined examination period, we expect to receive feedback from the students and staff. Bearing this in mind, we can improve the quality of the service and fix the different issues presented during the participants' interactions with this tool.