\chapter{Introduction}
% Objective:
% Develop a tool for the future students who will study nonlinear optics in fibres so that they can appreciate more easily the different physical processes happening to a short pulse propagating in a dispersive nonlinear medium. These include the influence of the dispersion (second and third order), the self-phase modulation, the emission of dispersive wave, the soliton dynamics. Eventually, the code will be released as a web-interface for the future students to directly modify the physical parameters of the pulse, but also of the fibre.**

The actual state of communications leads to new challenges in the field of optical systems, which causes an increased interest from students in optical communications. Nonlinear optics in fibers involve different physical processes while a short pulse propagates in a dispersive nonlinear medium. 

Despite the disturbing character of the nonlinear effects, e.g. crosstalk or increased bit error rates; one can take advantage of them in some applications due to their importance on telecommunications. Combinations of these effects with optical filters and interferometers can originate "purely optical regeneration",  or, together with the dispersion, solitons can be formed \cite{rein}.


Effects like the influence of second or third-order dispersion, self-phase modulation, the emission of a dispersive wave, or the soliton dynamics are some of the topics of this thesis. The main objective is to use the existing theory and create a code capable of simulating these effects and then deploy it online to provide an accessible way for students (or people interested in understanding the concepts surrounding nonlinear optics) to recapitulate the theory seen in the lecture. The third chapter explains the development and deployment of a tool using the created code. \textbf{**ADD Method of feedback**} 

At the same time, it is pertinent to understand some of the necessities of the students who will be the final users of the tool created. This interactive service should be able to adapt itself to be a good add-on and aid the professors with their explanations throughout the semester. Hence those who take the lectures can explore what usually is on the blackboard or the books and interact with plots and solutions of equations going further than just replacing values in an equation and waiting for it to be correct. Here is when Interactive Learning and its advantages obtain relevance and thus bestow meaning to the core of this project.

